\documentclass[a4paper,10pt]{article}
%\usepackage{geometry}                % See geometry.pdf to learn the layout options. There are lots.
%\geometry{landscape}                % Activate for for rotated page geometry
%\usepackage[parfill]{parskip}    % Activate to begin paragraphs with an empty line rather than an indent
\usepackage{graphicx}
\usepackage{amsmath}
\def\floor#1{\lfloor #1 \rfloor}
\def\ceil#1{\lceil #1 \rceil}
% \usepackage{amssymb}
\usepackage{epstopdf}
\usepackage[utf8]{inputenc}
\usepackage{titlesec}
\usepackage[titletoc]{appendix}
\titleformat{\chapter}[hang]{\bf\Huge}{\thechapter}{1cm}{}

\pagestyle{plain}
% -------------------- this stuff for code --------------------

\usepackage{anysize}
\marginsize{30mm}{30mm}{20mm}{20mm}

\newenvironment{formal}{%
  \def\FrameCommand{%
    \hspace{1pt}%
    {\color{blue}\vrule width 2pt}%
    {\color{formalshade}\vrule width 4pt}%
    \colorbox{formalshade}%
  }%
  \MakeFramed{\advance\hsize-\width\FrameRestore}%
  \noindent\hspace{-4.55pt}% disable indenting first paragraph
  \begin{adjustwidth}{}{7pt}%
  \vspace{2pt}\vspace{2pt}%
}
{%
  \vspace{2pt}\end{adjustwidth}\endMakeFramed%
}

\newenvironment{changemargin}[2]{\begin{list}{}{%
\setlength{\topsep}{0pt}%
\setlength{\leftmargin}{0pt}%
\setlength{\rightmargin}{0pt}%
\setlength{\listparindent}{\parindent}%
\setlength{\itemindent}{\parindent}%
\setlength{\parsep}{0pt plus 1pt}%
\addtolength{\leftmargin}{#1}%
\addtolength{\rightmargin}{#2}%
}\item }{\end{list}}

\usepackage{color}
\usepackage{dsfont}
\usepackage[bitstream-charter]{mathdesign}
\usepackage[scaled]{helvet}
\usepackage{inconsolata}


\definecolor{colKeys}{rgb}{0,0,0.9} 
\definecolor{colIdentifier}{rgb}{0,0,0} 
\definecolor{colString}{rgb}{0.7,0,0} 
\definecolor{colComments}{rgb}{0,0.6,0} 
\usepackage{listings}
\lstset{
  language=python,
  stringstyle=\color{colString},
  keywordstyle=\color{colKeys},
  identifierstyle=\color{colIdentifier},
  commentstyle=\color{colComments},
  numbers=left,
  tabsize=4,
  frame=single,
  breaklines=true,
  basicstyle=\small\ttfamily,
  numberstyle=\tiny\ttfamily,
  framexleftmargin=0mm,
  xleftmargin=7mm,
  xrightmargin=7mm,
  frameround={tttt},
  captionpos=b
}

%% Headers and footers
\usepackage[section]{placeins}
\newcommand{\clearemptydoublepage}{\newpage{\pagestyle{plain}\cleardoublepage}}
\DeclareMathOperator*{\argmax}{arg\,max}
\usepackage[T1]{fontenc}
\usepackage{enumerate}
\usepackage{afterpage,lastpage}
\usepackage[includeheadfoot,margin=2.5cm]{geometry}
\geometry{letterpaper}                   % ... or a4paper or a5paper or ... 

% -------------------- end of code stuff --------------------



\DeclareGraphicsRule{.tif}{png}{.png}{`convert #1 `dirname #1`/`basename #1 .tif`.png}

\makeatletter \def\thickhrulefill{\leavevmode \leaders \hrule height 1pt\hfill
\kern \z@}


\author{Paul Gribelyuk (pg1312, a5)}
\makeatother
\title{\Large \#DOC429 - Study Notes for Parallel Algorithms}
\date{\today}

\begin{document}
\maketitle

\section{Metrics (64 pages)}
\subsection{Architectures}
\begin{itemize}
\item \emph{Message passing} used in the case of many machines having only local memory.
\item \emph{Shared address space} used when multiple processors in same computer access same memory;  \emph{UMA} has equal access times for all, otherwise \emph{NUMA}, so address space is distributed among processors
\item \emph{Interconnect Network (IN)} provides hardware to pass messages; topology defines performance: ring, mesh, hypercube
\end{itemize}
\emph{PRAM} is an idealization of shared memory MIMD with UMA.  Different access modes:
\begin{itemize}
\item EREW - exclusive read exclusive write; minimizes concurrency
\item CREW - concurrent read exclusive write
\item CRCW - concurrent read, concurrent write; to write concurrently, semantics can be \emph{common}, \emph{arbitrary}, \emph{priority}, \emph{reduce} (sum or max or some other reduce operation).
\item ERCW - dumb
\end{itemize}
\subsection{Embedding}
\emph{Binary grey codes} used to convert ring network to hypercube:
\begin{eqnarray}
G(0,1) & = & 0 \\
G(1,1) & = & 1 \\
G(i, n+1) & = & \left\{ \begin{array}{ll}
G(i,n) & i < 2^n \\
2^n + G(2^{n+1} - 1 - i, n) & i \geq 2^n \end{array} \right.
\end{eqnarray}
Mesh to hypercube can be done by concatenating RGC of each dimension.  For node $i$ in $2^{r_1}\times\ldots \times2^{r_m}$ mesh, mapping is $G(i_1, r_1)G(i_2,r_2)\ldots G(i_m,r_m)$.  Can also map a tree to a hypercube.  At each level $k$ of the tree, assign the $k$-th bit either 0 or 1.

\subsection{Communication patterns}
\begin{itemize}
\item Simple message
\item One-to-All broadcast; dual is single node accumulation
\item All-to-All broadcast;  dual is multi-node accumulation
\item One-to-One personalized; dual is single node gather
\item All-to-All personalized scatter, dual is multi-node gather
\item Other (?)
\end{itemize}

\subsection{Performance}
\begin{itemize}
\item Run Time: $T_p$
\item Speedup: $S_p = \frac{\text{best serial } T}{T_p}$
\item Efficiency: $E_p = \frac{S_p}{p}$ is speedup per processor, usually less than 1
\item Cost: $C_p = p\cdot T_p$, total amount of computation done on $p$ processors.  Cost optimal if $E_p = \Theta(1)$
\item 
\end{itemize}
Cost optimality means costs are quivalent to best serial runtime! \\
Example:  to add $n$ numbers on hypercube with $p=2^d$ nodes each nodes doing $k=n/p$ serial steps, then partial sums reduced via single node accumulation:
\begin{eqnarray}
\text{best serial } T &=& n\\
T_p &=& \frac{n}{p} + \log{p} \\
S_p &=& \frac{p}{\frac{n}{p} + \log(p)} \\
E_p &=& \Theta(1/(n/p + \log(p))) \\
C_p &=& n + p\log(p)
\end{eqnarray}
So cost-optimal if $n = \Theta(p\log(p))$. \\
To understand how algorithm scales, we look at \emph{isoefficiency}.  $O_p = C_p - Work$ is a measure of communication latency.

\subsection{Communication Costs}
\begin{itemize}
\item \emph{startup time} $t_s$ incurred once per message
\item \emph{per-hop time} $t_h$ 
\item \emph{transfer time} $t_w$
\end{itemize}
Store and forward messages:
$$
t_{comm} = t_s + (mt_w + t_h) l
$$
Cut-through routing:
$$
t_{comm} = t_s + mt_w + lt_h
$$
\begin{itemize}
\item \emph{Diameter} is maximum length between any two nodes
\item \emph{Arc connectivity} is how many links must be broken to fragment network
\item \emph{Bisection width} is how many links must be broken to split network into 2 equal halves
\item \emph{cost} is usually the number of links in a network
\end{itemize}
\begin{tabular}{|c|cccc|}
\hline
topology & diameter & bisection & arc con & cost \\
\hline
completely connected & 1 & $p^2/4$ & p-1 & $p(p-1)/2$ \\
star & 2 & 1 & 1 & p-1 \\
binary tree & $2\log((p+1)/2$ & 1 & 1 & p-1 \\
linear array & $p-1$ & 1 & 1 & $p-1$ \\
2-D mesh w/o wrap & $2(\sqrt(p)-1)$ & $\sqrt(p)$ & 2 & $2p\sqrt(p-1)$ \\
2-D wraparound & $2\floor{\sqrt(p)/2}$ & $2\sqrt{p}$ & 4 & $2p$ \\
hypercube & $log{p}$ & $p/2$ & $log{p}$ & $p\log{p}/2$ \\
wrap k-ary d cube & $d\floor{k/2}$ & $2k^{d-1}$ & 2d & dp \\
\hline
\end{tabular}
\\
\\
Different costs associated with these topologies: \\ \\
\begin{tabular}{|c|ccc|}
\hline
operation & hypercube & mesh & ring\\
\hline
one-to-all bcast & $\min\left\{(t_s+t_w m)\log(p), 2(t_s\log(p) + t_w m)\right\}$ & $2(t_s + t_w m\log(p)$ & $t_s + t_w m\log(p)$ \\
all-to-all bcast &  $t_s\log(p) + t_w m(p-1)$ & $(t_s + t_w m\sqrt{p})(\sqrt{p}-1)$ & $(t_s+t_w m)(p-1)$ \\
all-reduce & $\min\left\{(t_s+t_w m)\log(p), 2(t_s\log(p) + t_w m)\right\}$ &  & \\
scatter, gather & $t_s log{p} + t_w  (p-1)$ & & \\
ATA personalized & $(t_s + t_w m)(p-1)$ & & \\
circular shift & $t_s + t_w m$ & & \\
\hline
\end{tabular}
\\
\\
\emph{Isoefficiency} is how $E$ scales with amount of work $W$ and with $p$.  Goal is to find a way to scale $W$ as a function of $p$.
$$
T_p = \frac{W + O(W, p)}{p} \quad\implies S_p = \frac{W}{T_p} = \frac{W\cdot p}{W + O(W,p)} \quad\implies E = \frac{S}{p} = \frac{1}{1 + O(W,p)/W}
$$
Another way to formulate cost-optimality:
\[pT_p = \Theta(W)\quad\implies W+O(W,p) = \Theta(W)\quad\implies W = \Omega(O(W,p))\]

\section{Dense Matrix (49 pages)}
\subsection{Matrix-Vector Multiply}
\section{Linear Equations (21 pages)}
\section{Partitioning (27 pages)}
\section{Search (73 pages)}
\section{MPI (31 pages)}

\end{document}

